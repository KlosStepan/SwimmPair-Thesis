\chapter{System design}
Reader will be familiarized with architecture of our application. There are two logical parts, \textbf{public web} and \textbf{private administration}. Private administration is hidden behing \textbf{login/password}. 
\par
When designing such system, object oriented approach and grouping of similar functions together is a must. There are objects that have to be moved around the web application described in previous chapter. These objects are Post, User, Club, Cup, Position and Region. Therefore we came up with a concept of managers. Each page of SwimmPair is composed of same headerer, menu, footer. The content part is filled with page's specific results of manager call used to construct data UI page layout. These managers are included and used in all pages via \textbf{start file}.
\section{Technologies}
Following technologies are used to implement SwimmPair application:
\begin{itemize}
    \item \textbf{HTML} is HyperText Markup Language \footnote{\citep{HTML5Standard}} - application pages are templated in HTML by PHP,
    \item \textbf{CSS} is Cascading Style Sheets \footnote{\citep{CSS3Standard}},
    \item \textbf{PHP} is a general-purpose scripting language geared toward web development \footnote{\citep{PHP74Standard}} - object model and backend services are provided by it,
    \item \textbf{JavaScript}  is a general-purpose scripting language that conforms to the ECMAScript specification \footnote{\citep{ECMADocu}},
    \item \textbf{MySQL} is an open-source relational database management system \footnote{\citep{MySQLDocu}},
    \item \textbf{Git} is a distributed version control system: tracking changes in any set of files - this project is versioned and kept in public GitHub repository \footnote{\url{https://github.com/KlosStepan/SwimmPair-Www}},
    \item \textbf{Docker} is a set of platform as a service products that use OS-level virtualization to deliver software in packages called containers \footnote{\citep{DockerDocu}} - used for deployment of out application,
    \item \textbf{Kubernetes} is an open-source container orchestration system for automating software deployment, scaling, and management \footnote{\citep{K8sDocu}} - used for production deployment of our application into cluster.
\end{itemize} 
\newpage
\section{Architecture overview}
Visitor comes to \textbf{app page}, where \textbf{managers} are included. From page there are API calls on Managers that retrieve and store data data as follows.
\newline
\includegraphics[scale=0.707]{img/app-schema.jpg}
\newline
\section{Model Managers}
Managers are written to provide API functionality for system administration. These managers are populating pages or taking new input from them and administer process of storing them. Each object has a manager handling it and accomodates database loads and stores controlled by transactions.
\begin{itemize}
    \item Cup / CupsManager
    \item User / UsersManager
    \item Club / ClubsManager
    \item Page / PagesManager
    \item Post / PostsManager
    \item Position / PositionsManager
    \item Region / RegionsManager
\end{itemize}
Managers are implemented to extract and store data of class by which they are named after.
\newpage
\section{User Interface mockups}
In this chapter we present UI mockups of both public and private parts of our applicaton. They serve as an initial virtualization mockups on which the real UI will be made. These mockups are not 1:1 guidances, rather an idea for reader and stakeholders in the beginning.
\newline
\includegraphics[scale=0.507]{img/def-U-Main.png}
\includegraphics[scale=0.507]{img/def-U-ListingCups.png}
\newline
\includegraphics[scale=0.507]{img/def-U-Cup.png}
\includegraphics[scale=0.507]{img/def-U-ListingUsers.png}
\newpage
Administration mockups
\newline
\includegraphics[scale=0.507]{img/A-administrace.png}
\includegraphics[scale=0.507]{img/A-new-cup.png}
\newline
\includegraphics[scale=0.507]{img/A-pairing.png}
\includegraphics[scale=0.507]{img/A-edit-page.png}
\newpage
\section{Database design}
For this purpose well defined database is a necessity. In this chapter we learn how the proposed objects are represented how relations between these objects are maintaned.
\newline
\includegraphics[scale=0.2175]{img/swimmpair_db_schema.png}
%\includegraphics[scale=0.45]{img/schema.png}
\newline

\iffalse
% Let's take PostsManager as an example. This manager handles Post and is implemented as follows.
\newline
\textbf{Object - Post.php}
\begin{lstlisting}
class Post
{
    public $id;
    public $timestamp;
    public $title;
    public $content;
    public $display_flag;
    public $author_user_id;
	public $signature_flag;

    public function __construct($id, $timestamp, $title, $content, $display_flag, $author_user_id, $signature_flag)

	//7/7: {id, timestamp, title, content, display_flag, author_user_id, signature_flag}
	public function Serialize()
}

\end{lstlisting}
\textbf{Manager - PostsManager.php} 
\begin{lstlisting}
class PostsManager
{
	private $mysqli;

    //Constructor - setting $mysqli to $this->mysqli
	public function __construct(mysqli $mysqli)

    //Handling functions retrieve/store  
	public function GetPostById($id)
  	public function FindLastNPosts($N)
	public function InsertNewPost($title, $content, $display_flag, $author, $signature_flag)
    public function UpdatePost($id, $title, $content, $display_flag, $signature_flag)

    //Private functions - auxiliary controller functions
	private function _CreatePostOrNullFromStatement(mysqli_stmt $statement)
	private function _CreatePostsFromStatement(mysqli_stmt $statement)
	private function _CreatePostFromRow(array $row)
}
\end{lstlisting}
\textbf{Demonstration - public function GetPostById(\$id)}
\begin{lstlisting}
public function GetPostByID($id)
{
	$statement = $this->mysqli->prepare("CALL `GetPostById`(?);");
	$statement->bind_param('i', $id);
	return $this->_CreatePostOrNullFromStatement($statement);
}
\end{lstlisting}
Objects and their managers are made in the same manner. They contain different set of data and different number public functions. Description is included in documentation chapter below. 
\fi
\iffalse
\section{Responsive layout}
Listed media queries are used to provide design of the web by manually overriding specific classes for desired user experience outcome.
\begin{itemize}
    \item Basic CSS design
    \item @media (max-width: 768px)
    \item @media (print)
\end{itemize}
\textbf{Basic CSS design} gives definition of colors and desktop layout of our application. \textbf{Media query with max-width: 768px} supports tablets and mobile devices while \textbf{media print} of cup pairing hides redundant controll and informative elements while it keeps the pairing of cup to be printed.
\fi
\iffalse
\section{Administrative tasks}
\begin{itemize}
    \item \textbf{Add Post/Edit Post} - from PostsManager call \newline InsertNewPost/UpdatePost
    \item \textbf{Approve Newly Registered Users} - swap flag \textbf{approved} to \textbf{1}
    \item \textbf{Pair Available Users On Cup Positions} - from UsersManager call UpdatePairing - calls several SQL Procs for different things in transaction and commits/rollbacks
    \item \textbf{Add User/Edit User} - from UsersManager call AddUser/UpdateUser
    \item \textbf{Add Cup/Edit Cup} - from CupsManager call AddCup/UpdateCup
    \item \textbf{Add Club/Edit Club} - from ClubsManager call AddClub/UpdateClub
    \item \textbf{Add Region/Edit Region} - from RegionsManager call AddRegion/UpdateRegion
    \item \textbf{Configure Stats Ordering} - delete ordering, insert ordering \textbf{Nth}-\textbf{statId} 
    \item \textbf{Edit Contacts} - from PagesManager call UpdatePage
\end{itemize} 
\section{Club Manager tasks}
\begin{itemize}
    \item \textbf{Add Cup} - from CupsManager call AddCup
    \item \textbf{Sign Up People From My Club As Available For Cup} - prihlasit\_moje\_lidi\_na.php then XMLHttpRequest/call\_update\_availability.php
\end{itemize}   
\section{Referee tasks}    
\begin{itemize}
    \item \textbf{Sign Myself As Available For Cup} - add my Id to table \textbf{cupId}-\textbf{userId}
\end{itemize}
\fi
\section {Administration tasks to API Functions}
This is table that shows what is used where.
\newline
\begin{tabularx}{0.9\textwidth} { 
  | >{\raggedright\arraybackslash}X 
  | >{\centering\arraybackslash}X 
  | >{\raggedright\arraybackslash}X | }
 \hline
 Add Post & system admin& InsertNewPost \\
 \hline
 Edit Post  & system admin  & UpdatePost  \\
 \hline
 Approve New Users & system admin & ApproveUser \\
 \hline
 Add User & system admin & AddUser \\
 \hline
 Edit User & system admin & UpdateUser \\
 \hline
 Add Club & system admin & AddClub \\
 \hline
 Edit Club & system admin & UpdateClub \\
 \hline
 Add Region & system admin & AddRegion \\
 \hline
 Edit Region & system admin & UpdateRegion \\
 \hline
 Configure Stats & system admin & UpdateOrdering \\
 \hline
 Edit Contacts & system admin & UpdatePage \\
 \hline
 Add Cup & club manager & AddCup \\
 \hline
 Sign Up People From My Club As Available For Cup & club manager & prihlasit moje lidi na.php then XMLHttpRequest/call update availability.php \\
 \hline
 Sign Myself As Available For Cup & referee & add my Id to table cupId-userId \\
\hline
\end{tabularx}