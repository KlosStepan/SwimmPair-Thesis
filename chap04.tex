\chapter{Testing}
There are two main ways to make sure that a web application works properly and fulfills its role. On one hand there is a code performance testing, performing test on backend level with dummy data insertion and performance benchmarking. On the other hand there is testing to assure that users are able to use system and to get inspiration for future UX improvements via SUS.
\section{Performance evaluation}
Performance script \textbf{dummy\_data\_benchmark.php} \footnote{In https://github.com/KlosStepan/SwimmPair-Www \textbf{dummy\_data\_benchmark.php}} is located in main swimmpair folder. It is performed on default database installation (with 2 admin users, with already existing clubs administered by application requesters, and with default referee positions).  
\newline
\textbf{The script has several tasks (tests) which are performed and benchmarked.}
\begin{enumerate}
    \item \underline{Create 98 Users} (no. 3-100) - each random affiliation to existing Club (no. 1-15).
    \item \underline{Create 12 Cups} - each random affiliation to existing Club (no. 1-15).
    \item \underline{Fetch new Users}, \underline{fetch new Cups} (+ \underline{fetch static Positions}).
    \item \underline{Create Availabilities} (20 Users available per Cup).
    \item \underline{Create Pairings} (each Availability gets 1 random position).
    \item \underline{Call stats queries} (20 - randomly either Clubs or Users stats w/ random club\_id or user\_id).
\end{enumerate}
\textbf{Docker Compose} - 2.3 GHz Core i5 (I5-8259U) RAM 16GB Storage 512GB
\newline
\begin{tabular}{ |c|c|c|c|c|c|c|c|c|c|c|c| } 
    \hline
    T/rep no. & \#1 & \#2& \#3& \#4& \#5& \#6& \#7& \#8& \#9& \#10 \\
    \hline
    Test \#1 & 7.02 & 7.04& 6.61& 7.89& 6.73& 6.62& 6.66& 7.34& 7.19& 6.54 \\ 
    Test \#2 & 0.08 & 0.06& 0.06& 0.06& 0.06& 0.10& 0.58& 0.60& 0.50& 0.15 \\ 
    Test \#3 & 7.02 & 7.05& 6.61& 7.90& 6.74& 6.63& 6.66& 7.35& 7.20& 6.55 \\ 
    Test \#4 & 1.24 & 1.05& 1.14& 1.05& 1.18& 1.17& 1.24& 1.15& 0.96& 1.59 \\ 
    Test \#5 & 8.02 & 7.87& 7.39& 8.95& 7.70& 7.81& 7.44& 8.12& 7.98& 7.36 \\ 
    Test \#6 & 1.29 & 1.10& 1.19& 1.12& 1.23& 1.22& 1.28& 1.19& 1.00& 1.62 \\ 
    TOTAL & \underline{9.31} & \underline{8.97}& \underline{8.59}&\underline{10.06}& \underline{8.93}& \underline{9.03}& \underline{8.72}& \underline{9.31}& \underline{8.98}& \underline{8.98} \\ 
    \hline
\end{tabular}
\newline
\textbf{Kubernetes} - \textbf{DOKS} Kubernetes v 1.25.4-do.0, s-1vcpu-2gb-intel
\newline
\begin{tabular}{ |c|c|c|c|c|c|c|c|c|c|c|c| } 
    \hline
    T/rep no. & \#1 & \#2& \#3& \#4& \#5& \#6& \#7& \#8& \#9& \#10 \\
    \hline
    Test \#1 & 7.08 & 6.87& 6.79& 6.85& 7.20& 7.10& 6.77& 6.85& 6.81& 6.79 \\ 
    Test \#2 & 0.04 & 0.03& 0.04& 0.04& 0.04& 0.03& 0.05& 0.05& 0.03& 0.04 \\ 
    Test \#3 & 7.08 & 6.88& 6.79& 6.86& 7.20& 7.10& 6.77& 6.85& 6.81& 6.79 \\ 
    Test \#4 & 0.84 & 0.59& 0.68& 0.81& 0.60& 0.55& 0.70& 0.82& 0.60& 0.67 \\ 
    Test \#5 & 7.72 & 7.40& 7.55& 7.56& 7.69& 7.60& 7.35& 7.56& 7.33& 7.40 \\ 
    Test \#6 & 0.86 & 0.61& 0.70& 0.84& 0.62& 0.57& 0.72& 0.84& 0.62& 0.69 \\ 
    TOTAL & \underline{8.57} & \underline{8.01}& \underline{8.25}& \underline{8.40}& \underline{8.31}& \underline{8.17}& \underline{8.07}& \underline{8.39}& \underline{7.95}& \underline{8.09} \\ 
    \hline
\end{tabular}
\section{System Usability Scale testing}
We carried on testing of our application by handing SUS questionare to 20 respondents. We then evaluated the scores in order to find out how our application stands. These people are are either managers or common referees \footnote{\cite{SUSDesc}}. 
\newline
\textbf{Questionare is made of 10 questions scored 1-5.}
\begin{enumerate}
    \item I think that I would like to use this system frequently.
    \item I found the system unnecessarily complex.
    \item I thought the system was easy to use.
    \item I think that I would need the support of a technical person to be able to use this system.
    \item I found the various functions in this system were well integrated.
    \item I thought there was too much inconsistency in this system.
    \item I would imagine that most people would learn to use this system very quickly.
    \item I found the system very cumbersome to use.
    \item I felt very confident using the system.
    \item I needed to learn a lot of things before I could get going with this system.
\end{enumerate}
\textbf{We calculated\footnote{((A1-1)+(5-A2)+(A3-1)+(5-A4)+(A5-1)+(5-A6)+(A7-1)+(5-A8)+(A9-1)+(5-A10))*2,5} SUS feedbacks based on responses from 20 people.}
\newline
\begin{tabular}{ |c|c|c|c|c|c|c|c|c|c|c|c| } 
    \hline
    Respondent / Q. no. & \#1 & \#2& \#3& \#4& \#5& \#6& \#7& \#8& \#9& \#10 \\
    \hline
    Petr A - \textbf{87.5}     & 5 & 1& 3& 1& 5& 1& 3& 1& 5& 2 \\ 
    Olga A - \textbf{72.5}     & 2 & 1& 3& 2& 4& 1& 5& 1& 3& 3 \\ 
    Marin H - \textbf{75}    & 3 & 1& 4& 2& 5& 1& 3& 2& 3& 2 \\ 
    Michaela H - \textbf{60} & 2 & 3& 3& 4& 3& 2& 4& 2& 5& 2 \\ 
    Stepan K - \textbf{85}   & 5 & 2& 4& 1& 3& 1& 3& 1& 5& 1 \\ 
    Matylda K - \textbf{80}  & 4 & 1& 4& 2& 4& 1& 4& 2& 4& 2 \\ 
    Lukas Kour. - \textbf{67.5} & 2 & 2& 5& 2& 4& 1& 4& 2& 2& 3 \\ 
    Jana K - \textbf{60}    & 1 & 2& 3& 2& 5& 2& 3& 2& 2& 2 \\ 
    Lukas Kous. - \textbf{92.5} & 5 & 1& 4& 1& 5& 1& 4& 1& 5& 2 \\ 
    Zuzana K - \textbf{70}   & 3 & 2& 5& 1& 3& 1& 3& 2& 3& 3 \\ 
    Eva K - \textbf{80}      & 3 & 2& 5& 1& 3& 1& 3& 1& 4& 1 \\ 
    Michael P - \textbf{75}  & 2 & 1& 4& 3& 4& 1& 4& 1& 3& 1 \\ 
    Lenka P - \textbf{70}    & 3 & 2& 5& 1& 3& 2& 3& 2& 3& 2 \\ 
    Daniela S - \textbf{77.5}  & 3 & 2& 5& 2& 3& 1& 4& 2& 4& 1 \\ 
    Magdalena S - \textbf{85} & 4 & 1& 4& 1& 4& 1& 5& 1& 3& 2 \\ 
    Jiri S - \textbf{62.5}     & 3 & 3& 5& 3& 4& 2& 3& 3& 2& 1 \\ 
    Hana S - \textbf{80}     & 2 & 2& 4& 1& 5& 1& 4& 1& 4& 2 \\ 
    Alena T - \textbf{90}    & 4 & 1& 5& 1& 3& 1& 4& 1& 5& 1 \\ 
    Magda Z - \textbf{85}    & 3 & 2& 5& 2& 4& 1& 4& 1& 5& 1 \\ 
    Vera Z - \textbf{75}     & 3 & 3& 4& 1& 3& 1& 3& 1& 4& 1 \\ 
    \hline
\end{tabular}
\newpage
So we plotted our tes results for further investivation. We can compare results and it's distribution in boxplot or look further into specific questions.
\begin{figure}[h]	
    \centering	
    \includegraphics[scale=0.257]{img/sus_general.png}
    \caption{Boxplot of 20 SUS score results.}
    \label{fig4.1:susgeneral}
\end{figure}

\begin{figure}[h]	
    \centering	
    \includegraphics[scale=0.256]{img/sus_breakdown.png}
    \caption{Breakdown of single questions results.}
    \label{fig4.2:susbreakdown}
\end{figure}
\newpage
\section{Unit testing}
Testing serves two puposes:
\begin{itemize}
    \item \textbf{firstly} - to ensure that functionality works as intended,
    \item \textbf{secondly} - not only that newly added functionaly works as intended but also that any previous functions and mechanisms were not broken either. 
\end{itemize}
There is bunch of PHPUnit tests for each Manager called ManagerTest.php in folder \textbf{tests/Unit} in our project. Tests check ordinary CRUD \footnote{CRUD = Create, Read, Update, Delete} functionalities.
\subsection{Local execution of tests}
Preview the results of test during local development by attaching VS Code to the running Container \footnote{Volume within \textbf{. : /var/www/html} is, in fact, our working folder.} and seeing results in PHP Tools by DEVSENSE \footnote{\url{https://www.devsense.com/en/features\#vscode}} on the left side. We can also open command line (within container environment already) \textbf{docker@6bd3d752da84:/var/www/html} and run tests simply by using \textbf{phpunit} command. 
\begin{figure}[h]	
    \centering	
    \includegraphics[scale=0.207]{img/phpunit_locally.png}
    \caption{Attached VS Code to container and running PHPUnit tests.}
    \label{fig4.3:phpunitlocally}
\end{figure}
\newpage
\subsection{GitHub Actions workflow}
We then load our project repository with folder with tests and proceed to let automated testing be done upon code push into the repository. After testing is done, we can see results of all steps and if nothing failed green checkbox gets added to our repository header next to hash as a bonus.   
\newline
\textbf{Steps that are crucial for our CI pipeline's QA\footnote{QA = quallity assurance} purpose:}
\begin{enumerate}
    \item code gets pushed into master brach of the repositor\footnote{\url{https://github.com/KlosStepan/SwimmPair-Www}},
    \item GitHub Action workload\footnote{\url{https://github.com/KlosStepan/SwimmPair-Www/blob/master/.github/workflows/main.yml}} gets triggered,
    \item creates database, fill it with dummy data, app connects to it,
    \item then \textbf{phpunit} command is run,
    \item \textbf{results are reported} for preview for us (\autoref{fig4.4:phpunitghactionswf}).
\end{enumerate} 
Result of all these steps are visible on GitHub website. All steps are openable for more deliberate investigation.
\begin{figure}[h]	
    \centering	
    \includegraphics[scale=0.205]{img/phpunit_actionswf.png}
    \caption{GitHub Actions workflows - steps of execution, phpunit step.}
    \label{fig4.4:phpunitghactionswf}
\end{figure}
\newline
Once the pipeline runs successfully, a checkbox appears next to the commit hash, providing a quick overview.
\begin{figure}[h]	
    \centering	
    \includegraphics[scale=0.435]{img/phpunit_actionscheck.png}
    \caption{Green checkbox next to the commit hash.}
    \label{fig4.5:phpunitghactionscheck}
\end{figure}
\iffalse
\newpage
Rest of this stuff mby \textbf{depr.}
\begin{lstlisting}
//1. Create 98 Users - random affil to 1-15
$usersManager->RegisterUser($first_name, $last_name, $email, 12345, $rights[$rights_idx], $ranks[$rrid_idx]->id, $clubs[$club_idx]->id);
//2. Create 12 Cups
$cupsManager->InsertNewCup($cups_names[$cup_name_idx]." ".rand(1, 8), "2023-".str_pad($j, 2, '0', STR_PAD_LEFT)."-26", "2023-".str_pad($j, 2, '0', STR_PAD_LEFT)."-28", $clubs[$club_idx]->id, $content[$content_idx]);
//3. Fetch new Users and Cups (&positions)
$users = $usersManager->FindAllActiveUsersOrderByLastNameAsc();
$cups = $cupsManager->FindAllUpcomingCupsEarliestFirst();
$positions = $positionsManager->FindAllPositions();
//4. Create Availabilities
$cupsManager->InsertNewAvailability($cups[$k]->id, $users[$user_idx[$kk]]->id, 1);
//5. Create Pairings (availabilities 1 random pos. for each)
$cupsManager->InsertNewPairing(($l+1), $positions[$position_idx]->id, $avails[$ll]->id);
echo("6. Call stats queries (20 either Clubs/Users stat queryings)<br/>\r\n");
$personCupsCount = $usersManager->CountCupsAttendanceOfUserGivenYear($users[$user_idx]->id, $year);
$stats_cups = $usersManager->CountOverallStatisticsOfUserGivenYear($users[$user_idx]->id, $year);
$stats_users = $usersManager->CountClubSeasonalStatistics($clubs[$club_idx]->id, $year);
\end{lstlisting}
This benchmark snippet probably \textbf{depr.}
\newline
\includegraphics[scale=0.707]{img/app-benchmarking.png}
\fi