%%% Bibliography (literature used as a source)
%%%
%%% We employ bibTeX to construct the bibliography. It processes
%%% citations in the text (e.g., the \cite{...} macro) and looks up
%%% relevant entries in the bibliography.bib file.
%%%
%%% The \bibliographystyle command selects, which style will be used
%%% for references from the text. The argument in curly brackets is
%%% the name of the corresponding style file (*.bst). Both styles
%%% mentioned in this template are included in LaTeX distributions.

\bibliographystyle{plainnat}    %% Author (year)
% \bibliographystyle{unsrt}     %% [number]

\renewcommand{\bibname}{Bibliography}

%%% Generate the bibliography. Beware that if you cited no works,
%%% the empty list will be omitted completely.

\bibliography{bibliography}

%%% If case you prefer to write the bibliography manually (without bibTeX),
%%% you can use the following. Please follow the ISO 690 standard and
%%% citation conventions of your field of research.

 \begin{thebibliography}{9}

 \bibitem{lamport94}
   {\sc Lamport,} Leslie.
   \emph{\LaTeX: A Document Preparation System}.
   2nd edition.
   Massachusetts: Addison Wesley, 1994.
   ISBN 0-201-52983-1.

 \bibitem{einstein} 
   Albert Einstein. 
   \textit{Zur Elektrodynamik bewegter K{\"o}rper}. (German) 
   [\textit{On the electrodynamics of moving bodies}]. 
   Annalen der Physik, 322(10):891–921, 1905.

 \end{thebibliography}
