\chapter*{Introduction}
\addcontentsline{toc}{chapter}{Introduction}
\par
As someone born in the mid-1990s, I've had a front-row seat to the evolution of personal computing and the rise of the internet. Even as a three-year-old, I was captivated by my father's first computer, which ran Windows 98. By the age of five, I already knew I wanted to be a programmer when I grew up, after realizing that I could write code and create public websites. Over time, I became increasingly fascinated by the stories of technology giants like Microsoft and Apple, who brought computers to our desks and smartphones to our pockets. Of course, my interest in the IT world runs much deeper than this brief overview can convey.
\subsection*{Why web applications}
\par
The dot-com bubble crash in the early 2000s marked a necessary correction of the overhyped optimism surrounding new technologies, helping the industry as a whole to mature. During that time, humanity experimented with several good ideas, created proofs of concepts, and laid new foundations for the industry \footnote{\cite{RFDis}}. However, it was the financial crisis of 2008 that truly opened up opportunities in the web space. While it left the average American customer poorer, it also spurred a new trend of money-saving services designed to help people cut costs or earn extra cash. Suddenly, instead of calling a taxi, people could use Uber to find a ride from an independent driver. And if they had a spare room, they could list it on Airbnb to make some extra money. Meanwhile, the lack of trust in the banking industry and monetary policy gave rise to Bitcoin and other cryptocurrencies. While some of these innovations may seem technically complex, a skilled software engineer could deploy a minimum viable product of each one in just a matter of weeks or months.
\subsection*{Motivation}
\par
I designed this thesis as a full-stack web application for a friend who works as a chief swimming referee and club manager, in order to help him save time for more important tasks. The process of developing this application has been invaluable training for me, as it required me to devise a solution to a problem that was similar in some ways to the minimum viable products mentioned earlier. Throughout this project, I gained a wealth of experience, insights, and lessons that I hope to apply in my future endeavors and career. As I see it, software engineering (in general \& with some good introductory practices \footnote{\cite{MNSWE}}) is a crucial craft for effecting positive change in the contemporary world. I was also motivated by a course teaching basics of web development at Faculty of Mathematics and Physics at Charles University\footnote{Web Applications - NSWI142, academic year 2018/2019 by doc. RNDr. Martin Kruliš, Ph.D. \& Mgr. Jan Michelfeit, \url{https://is.cuni.cz/studium/predmety/index.php?do=predmet&kod=NSWI142}} despite doing a lot of webdev on my own as well. Building new things is an exciting adventure that holds endless possibilities.
\par
Software engineering is a crucial craft for effecting positive change in the contemporary world, as it enables us to create new tools, platforms, and experiences that make people's lives easier, more productive, and more fulfilling. In a sense, building things has become a modern adventure, filled with opportunities to innovate, explore, and improve the world around us.